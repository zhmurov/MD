\documentclass[12pt,a4paper]{article}

%\setlength{\textheight}{240mm} %A4
\setlength{\textheight}{230mm} %letter
\setlength{\oddsidemargin}{-10mm}
\setlength{\evensidemargin}{0mm}
%\setlength{\textwidth}{180mm} %A4
\setlength{\textwidth}{185mm} %letter
\setlength{\topmargin}{-15mm}

\usepackage[utf8]{inputenc}
%\usepackage[russian]{babel} %if you need russian language

\usepackage{color,xcolor}
\usepackage{amssymb,amsthm,amsmath,amsfonts,latexsym}
\usepackage{graphics}
\usepackage{graphicx}
\usepackage{setspace}
\usepackage[square,numbers,sort&compress]{natbib}
\usepackage{verbatim}
\usepackage[font=small,labelfont=bf,width=16cm]{caption}
\usepackage{indentfirst}
\usepackage{subcaption}

\newcommand{\degree}{\ensuremath{^\circ}}

\setlength{\bibsep}{0.0in}
%\linespread{1.5}

\captionsetup{width=\textwidth}

\begin{document}

\section{Installation}
\label{sec:inst}
\noindent
To get the code you can download archive or execute the following commands in the terminal:
\begin{verbatim}
    git clone https://github.com/zhmurov/MD
\end{verbatim}
To compile, run `make' from the code directory:
\begin{verbatim}
    cd MD/src/
    make
\end{verbatim}
This will produce `mrg-cg2' executable which is the program. To run simulations, first make sure that you have had topology, coordinate, parameter and configure file. You should pass the configuration file as the first parameter to the program (the description of this file follows):
\begin{verbatim}
    <instalation-path>/MD/bin/mrg-cg2 <config_file>.conf
\end{verbatim}

\section{Configuration file}

\subsection{Input}

{\bf device} ---  ID of NVidia card to run simulations on. Use “nvidia-smi” or “deviceQuery” from NVidia SDK to check devices ids.

{\bf topology\_file} --- path to the structure topology file. Format "*.top".

{\bf parameter\_file} --- path to the interaction parameters file (DNA-Ions, DNA-Protein, Protein-Ions). Includes potential parameters: BONDSCLASS2ATOM, ANGLECLASS2, GAUSSEXCLUDED.

{\bf coordinates\_file} ---  path to the structure initial coordinates file. Format "*.xyz".

{\bf structure\_file} --- path to the file with parameters of fixed atoms. Format "*.pdb".

\subsection{Calculation}

{\bf timestep} --- time-scale of one simulation step. Units: ps.

{\bf run} --- number of simulation steps, natural number. 

{\bf temperature} --- temperature. Units: K.

{\bf seed} --- random seed, determines a random spread of the initial velocities of the system.

{\bf pbc\_xlo} --- the minimum x coordinate of the box of periodically specified conditions. Units: nm.

{\bf pbc\_ylo} --- the minimum y coordinate of the box of periodically specified conditions. Units: nm.

{\bf pbc\_zlo} --- the minimum z coordinate of the box of periodically specified conditions. Units: nm.

{\bf pbc\_xhi} --- the maximum x coordinate of the box of periodically specified conditions. Units: nm.

{\bf pbc\_yhi} --- the maximum y coordinate of the box of periodically specified conditions. Units: nm.

{\bf pbc\_zhi} --- the maximum z coordinate of the box of periodically specified conditions. Units: nm.


{\bf fix\_momentum} --- conserve rotational momentum of the system
\begin{itemize}
\item fix\_momentum\_freq --- frequency with which the rotational momentum will be zeroed
\end{itemize}


{\bf integrator} --- switch on one of the following integrators:
"leap-frog", "velocity-verlet", "leap-frog-nose-hoover", "leap-frog-new", "leap-frog-overdumped", "steepest-descent".
\begin{itemize}
\item nose-hoover-tau --- nose-hoover thermostat parameter.
\item nose-hoover-T0 --- nose-hoover thermostat parameter.
\item max\_force --- steepest-descent parameter.
\end{itemize}

{\bf langevin} --- switch on the Langevin dynamics.
\begin{itemize}
\item langevin\_seed --- parameter of the random velocity distribution in the Langevin dynamics.
\item damping --- Langevin parameter.
\end{itemize}

\subsection{Potentials}

{\bf fene} --- switch on FENE potential (amino acid bond in one thread). Switches: "on/off".  NECESSARY for protein calculations.

{\bf lennardjones} --- switch on Lennard-Jones potential. Switches: "on/off". NECESSARY for protein calculations.

{\bf repulsive} --- switch on repulsive potential. Switches: "on/off". NECESSARY for protein calculations, But is not needed in the calculations of the general DNA-protein model.
\begin{itemize}
\item rep\_eps --- the depth of the potential well.
\item rep\_sigm --- zero interaction energy distance.
\end{itemize}

{\bf bondclass2atom} --- switch on bond interaction along the DNA chain. Switches: "on/off". NECESSARY for protein calculations. Potential parameters are in parameter file.

{\bf angleclass2} --- switch on angle interaction along the DNA chain. Switches: "on/off". NECESSARY for protein calculations. Potential parameters are in parameter file.

{\bf gaussexcluded} --- switch on interactions between DNA, ions and protein. Switches: "on/off". NECESSARY for protein calculations. Potential parameters are in parameter file.
\begin{itemize}
\item pairs\_cutoff --- radius of a small pairs list (nm).
\item pairs\_freq --- frequency of the small pairs list update.
\item possible\_pairs\_cutoff --- radius of a big pairs list (nm).
\item possible\_pairs\_freq --- frequency of the big pairs list update.
\end{itemize}

{\bf pppm} --- switch on $P^3M$ (Particle-Particle-Particle-Mesh) method. Switches: ``on/off''. NECESSARY for protein calculations.
\begin{itemize}
\item pppm\_order --- $P^3M$ order. Default value: 5.
\item pppm\_accuracy --- $P^3M$ accuracy.
\end{itemize}

{\bf coulomb} --- switch on the electrostatic potential. Switches: ``on/off''. NECESSARY for protein calculations. Potential parameters are in parameter file.
\begin{itemize}
\item dielectric --- permittivity, real. Default value: 1.0.
\item nb\_cutoff --- nonbonded interaction distance (nm).
\item coul\_cutoff --- Coulomb potential interaction distance (nm).
\end{itemize}

{\bf push\_sphere} --- switch on pushing sphere potential. Switches: "on/off".
\begin{itemize}
\item ps\_center\_of\_sphere --- pushing sphere center coordinates.
\item ps\_radius0 --- pushing sphere initial radius.
\item ps\_radius --- pushing sphere final radius.
\item ps\_update\_freq --- frequency of sphere coordinates update.
\item ps\_harmonic --- set compression potential as harmonic. Switches: "on/off". If disabled, Lennard-Jones potential is used by default.
\item ps\_sigma --- the depth of the potential well for repulsive Lennard-Jones potential. 
\item ps\_epsilon --- coefficient of restitution (in case of harmonic potential) / zero interaction energy distance (in case of repulsive Lennard-Jones potential)
\item ps\_mask --- switch on particles compression. Switches: "on/off". By default, only DNA particles are compressed.
\item ps\_mask\_pdb\_filename --- containing compressible atoms parameters file. Format "*.pdb".
\end{itemize}

{\bf fixation} --- switch on particle fixation. Switches: "on/off".
\begin{itemize}
\item fix\_atomtype --- type of fixed particles (1 -- DNA, 2 -- $Na^+$, 3 -- $Cl^-$, 4 -- protein)
\end{itemize}

{\bf indentation} --- switch on indentation potential. Switches: "on/off".
\begin{itemize}
\item ind\_tip\_radius --- radius of the virtual sphere representing cantilever tip (nm).
\item ind\_base\_coord --- current cantilever base coordinates (nm)
\item ind\_tip\_coord --- current cantilever tip coordinate (nm)
\item ind\_base\_freq --- the frequency of cantilever base displacement by ind\_vel.
\item ind\_n --- substrate cantilever tip normal vector (vector).
\item ind\_vel --- the value define the displacement of the virtual particle, representing cantilever base, every ind\_base\_freq steps. Actual cantilever base velocity can be calculated as $ind\_vel/(ind\_base\_freq -- timestep)$
\item ind\_ks --- spring constant of the cantilever (kJ/mol nm$^2$).
\item ind\_eps --- repulsive energy factor for Lennard-Jones potential (kJ/mol).
\item ind\_sigm --- repulsive distance for the Lennard-Jones potential. Note that potential is shifted to the surface of the cantilever tip sphere (nm).
\item sf\_coord --- current surface coordinate (nm).
\item sf\_n --- substrate surface normal vector (vector).
\item sf\_eps ---  repulsive energy factor for the surface Lennard-Jones potential (kJ/mol). 
\item sf\_sigm --- repulsive distance for the surface Lennard-Jones potential (nm).
\end{itemize}

\subsection{Output}

{\bf psf\_filename} --- the path to save the topology. Format: "*.psf".

{\bf dcd\_filename} --- filename to write coordinates output. Format: "*.dcd".

{\bf dcd\_freq} --- frequency of writing out structure coordinates in .dcd output file in course of simulation. Natural number.

{\bf energy\_filename} --- name of the output file to save resulted energy. Format: "*.dat".

{\bf energy\_freq} --- frequency of writing out energy in output file in course of simulation. Natural number.

{\bf output\_xyz} --- switch on final coordinates saving. Switches: "on/off".

{\bf output\_xyz\_file} --- filename to write final coordinates output. Format "*.xyz".

{\bf ps\_presure\_filename} --- filename to write pushing sphere pressure output. Fromat: "*.dat".

{\bf pdb\_cant\_filename} --- name of the output file with the position of cantilever tip. Format: "*.pdb".

{\bf dcd\_cant\_filename} --- name of the output file with the coordinates of cantilever tip. Format: "*.dcd".

\subsection{Example}

Consider an example of configuration file. There is equilibration simulation of dna system for 1 ns trajectory.

\begin{comment}
\begin{table}[h!]
\centering
\caption{equil.conf}
\begin{tabular}{l l}
\hline
device & 2 \\
name & dna \\
topology\_file & <name>.top \\
parameter\_file & par.dat \\
coordinates\_file & <name>.xyz \\ \hline
timestep & 0.005 \\
run & 200000 \\
temperature & 300 \\
seed & 5464754 \\
integrator & leap-frog-nose-hoover \\
nose-hoover-tau & 100 \\
nose-hoover-T0 & 300 \\
pbc\_xlo & -20 \\
pbc\_ylo & -20 \\
pbc\_zlo & -20 \\
pbc\_xhi & 20 \\
pbc\_yhi & 20 \\
pbc\_zhi & 20 \\ \hline
bondclass2atom & on \\ 
angleclass2 & on \\
gaussexcluded & on \\
pairs\_cutoff & 4 \\
pairs\_freq & 10 \\
possible\_pairs\_cutoff & 7 \\
possible\_pairs\_freq & 100 \\
pppm & on \\
pppm\_order & 5 \\
pppm\_accuracy & 0.00001 \\
coulomb & on \\
dielectric & 80.0 \\
nb\_cutoff & 1.5 \\
coul\_cutoff & 3 \\ \hline
psf\_filename & <name>\_equil.psf \\
dcd\_filename & <name>\_equil.dcd \\
dcd\_freq & 20 \\
energy\_filename & <name>\_out\_equil.dat \\
energy\_freq & 20 \\
output\_xyz\_file & <name>\_equil.final.coord.xyz \\ \hline
\end{tabular}
\label{tab:example}
\end{table}
\end{comment}

\begin{verbatim}
device                     2           # NVidia GPU device index
\end{verbatim}
Input files, including topology file in Gromacs format (.top), LAMMPS-style .dat file with parameters for Bond Class 2, Angle Class 2 and Gauss-excluded potentials and coordinate file in .xyz format.
\begin{verbatim}
name                       dna 
topology_file              <name>.top 
parameter_file             par.dat 
coordinates_file           <name>.xyz  
\end{verbatim}
Parameters for integrator:
\begin{verbatim}
timestep                   0.005        # 5 fs time step
run                        200000       # 1 ns trajectory 
temperature                300 
seed                       5464754 
integrator                 leap-frog-nose-hoover 
nose-hoover-tau            100 
nose-hoover-T0             300 
\end{verbatim}
Box with boundary conditions:
\begin{verbatim}
pbc_xlo                    -20          # Low coordinate x of the box
pbc_ylo                    -20          # Low coordinate y of the box 
pbc_zlo                    -20          # Low coordinate z of the box 
pbc_xhi                     20          # High coordinate x of the box
pbc_yhi                     20          # High coordinate y of the box
pbc_zhi                     20          # High coordinate z of the box
\end{verbatim}
Potentials:
\begin{verbatim}
bondclass2atom              on  
angleclass2                 on 
gaussexcluded               on 
pairs_cutoff                4 
pairs_freq                  10 
possible_pairs_cutoff       7 
possible_pairs_freq         100 
pppm                        on 
pppm_order                  5 
pppm_accuracy               0.00001 
coulomb                     on 
dielectric                  80.0 
nb_cutoff                   1.5 
coul_cutoff                 3
\end{verbatim}
Output:
\begin{verbatim}
psf_filename                <name>_equil.psf 
dcd_filename                <name>_equil.dcd 
dcd_freq                    20 
energy_filename             <name>_out_equil.dat 
energy_freq                 20 
output_xyz					on
output_xyz_file             <name>_equil.final.coord.xyz  
\end{verbatim}

\end{document}

